\documentclass{article}

\usepackage{amsmath}  % For \DeclareMathOperator
\usepackage{siunitx}

\def\simpledef{x}
\def\defcontainingcontrolsequence{\simpledef + y}
\def\defwithargs#1#2{#1 + #2}
\def\defcontainingstylecontrolsequence{\mathbf x}
\newcommand{\commandwithargs}[1]{#1}
\DeclareMathOperator{\op}{op}
\def\nesteddefwitharg#1{\mathbf #1}
\def\parentdef{\nesteddefwitharg{xx}}
\newcommand{\renewed}{x}
\renewcommand{\renewed}{x}

\begin{document}

% For expected behavior of TeX when reading spaces, see Chapter 7 of the TeXBook
% Can use '\relax' as a noop.

$\num{10000}$

% There should be no log output for this control sequence, as it appears outside 
% of a math environment.
\simpledef

% Simple case:
% When you see a def, look at the token for it.
% Grab the token and all of its arguments. See what gets queued into the gullet
% during the expansion. That is the expansion.
$\simpledef$

$\defcontainingcontrolsequence$

$\defcontainingstylecontrolsequence$

$\defwithargs{x}{y}$

$\defwithargs {x}{y}$

% This is a tricky case to handle. Need to support iterative expansion.
$\defwithargs{\simpledef}{y}$

$\commandwithargs{x}$

$\op x$

$\parentdef$

\end{document}
